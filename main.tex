\documentclass[UKenglish]{beamer}


\usetheme{CambridgeUS}


\usepackage[utf8]{inputenx} % For æ, ø, å
\usepackage{csquotes}       % Quotation marks
\usepackage{microtype}      % Improved typography
\usepackage{amssymb}        % Mathematical symbols
\usepackage{mathtools}      % Mathematical symbols
\usepackage[absolute, overlay]{textpos} % Arbitrary placement
\setlength{\TPHorizModule}{\paperwidth} % Textpos units
\setlength{\TPVertModule}{\paperheight} % Textpos units
\usepackage{tikz}
\usepackage{biblatex}

\usetikzlibrary{overlay-beamer-styles}  % Overlay effects for TikZ


\author{Igar Tarasyuk}
\title{Safety and efficacy of the ChAdOx1 nCoV-19 vaccine (AZD1222) against SARS-CoV-2}
\subtitle{Usage of the theme \texttt{CambridgeUS}}
\addbibresource{ChAd_library.bib}

\begin{document}

\nocite{*}


\section{}
\begin{frame}{}
    \includegraphics[width= \textwidth]{Images/earlyrhythmcontrol.PNG}
    \Caption{A Journal Club Article}
\end{frame}
% Use
%
%     \begin{frame}[allowframebreaks]{Title}
%
% if the TOC does not fit one frame.
\begin{frame}{Agenda}
    \tableofcontents
\end{frame}


\section{Current Management}
\subsection{Heart Foundation Guidelines (2018)}


\begin{frame}{National Heart Foundation of Australia and the Cardiac Society of Australia and New Zealand: Australian Clinical Guidelines for the Diagnosis and Management of Atrial Fibrillation 2018}
    \begin{alertblock}{Arrhythmia management - rhythm control versus rate control}
        A rhythm or rate control should be selected in discussion with the patient.
    \end{alertblock}

    \begin{alertblock}{Stroke Prevention - predicting stroke risk}
        The sexless CHA2DS2-VA score is recommended for predicting stroke risk in AF.
    \end{alertblock}
    
    \begin{alertblock}{Stroke prevention - anticoagulation}
        Oral anticoagulant therapy is recommended in patients with non-valvular AF whose VA-Score is 2 or more.
    \end{alertblock}
        
\end{frame}

\begin{frame}{Rate and Rhythm}
    %\begin{alertblock}{Arrhythmia management - acute rate control}
        %Beta antagonists, or CCBs are recommended for acute control. Amiodarone can be used in highly symptomatic patients or with LV dysfunction. Digoxin can be used as add-on, or if other therapies are contraindicated.
    %\end{alertblock}
    
    \begin{alertblock}{Arrhythmia management - long-term rate control}
        Continue beta antagonists or CCBs, monitor amiodarone or digoxin therapy. Rate control should not be used with rhythm control. Consider AV nodal ablation if this strategy fails.
    \end{alertblock}
        
%\end{frame}

%\begin{frame}{Rhythm Guidance}
   % \begin{alertblock}{Arrhythmia management - acute rhythm control}
       % Either cardioversion emergently if haemodynamically unstable, or if pharmocological intervention fails. Flecainide with an additional AV nodal blocking agent. Intravenous Amiodarone may also be considered.
    %\end{alertblock}
    
    \begin{alertblock}{Arrhythmia management - long-term rhythm control}
        Anything from Flecainide, Amiodarone, Sotalol, or Beta andrenoceptor antagonist can be considered for maintenance of sinus rhythm.
    \end{alertblock}
        
\end{frame}

\begin{frame}{AF Ablation}
    \begin{alertblock}{Arrhythmia management - percutaneous catheter AF ablation}
        Catheter ablation can be considered in select patients with symptoms, heart failure; and in those with refractory AF to Class I or III antiarrhythmics.
    \end{alertblock}
    
   

\end{frame}
    
\subsection{Current Evidence}

\begin{frame}{AFFIRM Trial}
    \begin{figure}
        \caption{All cause mortality outcomes amongst 4060 patients randomised to either rhythm or rate control strategy}
        \includegraphics[width= 0.6\textwidth]{Images/AFFIRM_rateandrhythm.PNG}
    \end{figure}
    
\end{frame}


\section{EAST-AFNET 4}

\begin{frame}{EAST-AFNET 4}
    An investigator-initiated, parallel-group, randomised, open, blinded-outcome-assessment trial.
    
    \begin{alertblock}{Study Question}
        \textit{"...test whether a strategy of early rhythm-control therapy that includes atrial fibrillation ablation would be associated with better outcomes in patients with early atrial fibrillation..."}
    \end{alertblock}
\end{frame}
    

\subsection{Study Design}


\begin{frame}[allowframebreaks]{Study Design}
   
    
    \begin{alertblock}{Population}
        Enrolled adults diagnosed with atrial fibrillation at most 12 months before enrollment who were over 75 with a history of TIA/CVA; OR had at least 2 of age over 65, female parts, heart failure, hypertension, diabetes, severe CAD, CKD3/4, LV hypertrophy.
    \end{alertblock}
    
    \begin{alertblock}{Intervention - Early Rhythm Control}
        Received standard of care in addition to early rate control which was some combination of antiarrhythmic drug, cardioversion, and AF ablation as selected by study site. Cohort were also provided single lead ECG monitors to transmit to site team, trigger adaptation of antiarrhythmic therapy.
    \end{alertblock}
    
    \begin{alertblock}{Control - Standard of Care and Rate Control}
        Were treated with standard of care and were not commenced on rhythm-control at enrollment.
    \end{alertblock}
    
    \begin{alertblock}{Primary Outcomes}
        Two primary outcome measures; first being composite of death from cardiovascular, cerebrovascular cause, or worsening of heart failure, or ACS. Second being number of inpatient nights per year.
    \end{alertblock}
    
        
    \begin{alertblock}{Follow-Up}
        All patients were followed-up at 2 years following enrollment, with a loss rate of 9 and 6.6 \% between the intervention and control arms respectively.
    \end{alertblock}
    
\end{frame}

\subsection{Recruitment}

\begin{frame}{Population and Randomisation}
    \includegraphics[width= \textwidth]{Images/allocation.PNG}

\end{frame}

\begin{frame}{Patient Demographics}
    \includegraphics[width= \textwidth]{Images/demographics.PNG}
\end{frame}

\begin{frame}{Inclusion Criteria}
    \includegraphics[width= \textwidth]{Images/criteria.png}
\end{frame}

\subsection{Intervention}

\begin{frame}[allowframebreaks]{Antiarrhythmic Therapy}
    \begin{itemize}
        \item Flecainide, Amiodarone, AF Ablation
        \item Propafenone - equivalent to Flecainide in European guidelines 
        \item Dronedarone - straddles treatment algorithms between first-line antiarrhythmics and Amiodarone
        \item "Other Antiarrhythmics"
    \end{itemize}
    
    \begin{figure}
        \includegraphics[width= \paperwidth]{Images/intervention.PNG}
        \Caption{Distribution of therapy between intervention (left) and control (arms). Additionally, 15.4\% of intervention arm patients required urgent re-evaluation of antiarrhythmic by the study group}
    \end{figure}
    
\end{frame}

\subsection{Control}

\begin{frame}[allowframebreaks]{Standard of Care}
    What treatment did the majority of patients receive?
    
    \begin{itemize}
        \item Oral Anticoagulant either DOAC or VKA
        \item Beta adrenoceptor antagonists - first line rate control amongst guidelines
        \item ACEi or ARB - mentioned in European guidelines as considered concurrent therapy with antiarrhythmics for preventions
    \end{itemize}
    
    Lesser therapies
    
    \begin{itemize}
        \item Diuretics and Statins - latter mentioned in ESC guidelines for prevention of remodelling
        \item Digoxin - considered as distal-line rate control in AUSNZ guideline.
    \end{itemize}
    \begin{figure}
        \includegraphics[width= \paperwidth]{Images/therapy.PNG}
    
    \end{figure}
    
\end{frame}

\section{Results}
\subsection{Primary Outcomes}

\begin{frame}{First Primary Outcome}
    \begin{figure}
        \includegraphics[width= \textwidth]{Images/primaryoutcome.PNG}
        \caption{Primary outcome results for significant events, notable risk reduction in death from cardiovascular disease.\\
        Additionally second primary outcome showed nil difference in number of inpatient nights per year 5.8 and 5.1 days between intervention and control arms.}
    \end{figure}
    
\end{frame}


\begin{frame}{Cumulative}
    \begin{figure}
        \includegraphics[width= 0.6\textwidth]{Images/cumulativeoutcome.PNG}
        \caption{Cumulative risk of first primary outcome between intervention and control arms.}
    \end{figure}
    
    
\end{frame}

\subsection{Secondary Outcomes}
\begin{frame}{Qualitative Outcomes}
    \begin{figure}
        \centering
        \includegraphics[width= \textwidth]{Images/secondaryoutcome.PNG}
        \caption{Secondary outcome measures demonstrating predominantly benefit in maintaining sinus rhythm for rhythm-control strategy, minimal other symptomatic benefit}
    \end{figure}
\end{frame}

\subsection{Hazard Outcomes}
\begin{frame}{Primum Nocere}
    \begin{figure}
        \centering
        \includegraphics[width= \textwidth]{Images/hazardoutcomes.PNG}
        \caption{Non-significant difference in all cause mortality between intervention and control arms. Overall rhythm-control strategy had higher rates of drug toxicity, procedural complications, and hospitalisations due to atrial fibrillation.}
    \end{figure}

\end{frame}

\section{Analysis}
\subsection{Critique}

\begin{frame}{Strengths}
    \begin{itemize}
        \item Statistical power
        \item Real world design
        \item Applicable patient cohort
    \end{itemize}
\end{frame}

\begin{frame}{Weaknesses}
    \begin{itemize}
        \item Lack of blinding to intervention
        \item Cumulative outcomes of mortality of clinical measurements
        \item Lacking proscribed standard of care  
        \item Adaptive anti-arrhythmic not properly controlled for
    \end{itemize}
\end{frame}

\subsection{Impact}

\begin{frame}{Is this significant?}
    
     \begin{quote}
        Our consensus is that the primary indication for catheter ablation of AF is the presence of symptomatic AF that is refractory or intolerant to at least one Class 1 or Class 3 antiarrhythmic medication.
    \end{quote}
    \Caption{National Heart Foundation of Australia consensus statement on catheter ablation as a therapy for atrial fibrillation (2013)}
    
\end{frame}



\section{References}


\begin{frame}[allowframebreaks]{References}

    \printbibliography
    
\end{frame}


\end{document}